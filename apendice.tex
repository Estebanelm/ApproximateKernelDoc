\chapter{Implicaciones sociales y éticas del proyecto}

El presente proyecto busca aportar al área de investigación en
redes neuronales convolucionales. Este tipo de redes neuronales
es utilizado principalmente para el reconocimiento y clasificación
de imágenes. Esto implica que se puede determinar el contenido
de una imagen de la misma forma que el cerebro humano procesa
e identifica a la misma imagen.

Con el incremento en la generación de datos por medio
de múltiples dispositivos y la creciente tendencia de el "internet de las cosas",
existe una creciente responsabilidad de parte de la sociedad de utilizar
las redes neuronales convolucionales de forma beneficiosa para esta.
Sociedades actuales hacen uso de este tipo de aplicaciones como
forma de control de masas o para aumentar ganancias monetarias de
poderosas empresas.

Sin embargo, también existen usos positivos para el reconocimiento
de imágenes. En aplicaciones médicas, aeroespaciales, tecnológicas,
científicas
y más, es posible ver los beneficios que las redes neuronales tienen
para la sociedad en general. Se espera que este proyecto represente
un avance que pueda ayudar a mejorar la vida de las personas.