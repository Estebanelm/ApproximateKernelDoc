\chapter{Conclusions and recommendations}

Las conclusiones no son un resumen de lo realizado sino a lo que ha llevado el
desarrollo de la tesis, no perdiendo de vista los objetivos planteados desde
el principio y los resultados obtenidos.  En otras palabras, qué se concluye o
a qué se ha llegado después de realizado la tesis de maestría.  Un error
común es ``concluir'' aspectos que no se desarrollaron en la tesis, como
observaciones o afirmaciones derivadas de la teoría directamente.  Esto último
debe evitarse.

Es fundamental en este capítulo hacer énfasis y puntualizar los
aportes específicos del trabajo.

Es usual concluir con lo que queda por hacer, o sugerencias para mejorar los
resultados.


\section{Jennier}

En este capítulo se presenta un resumen con los descubrimientos más
relevantes del capítulo de resultados, mediante las conclusiones y
recomendaciones generales obtenidas a lo largo del desarrollo del
proyecto.
Debe reflejar de forma contundente y ordenada si se lograron los objetivos
propuestos y si se cumplieron los entregables que se habían definido en el
anteproyecto. Se deben incluir también los aspectos en que no se ahondó
en el estudio pero que se consideran relevantes y quedaron como trabajos
futuros.

