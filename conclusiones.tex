\chapter{Conclusions and recommendations}

In the next section, the main discoveries of the project are
presented. Also, some recommendations are given on what
future projects can focus on and how the current work
can be expanded.

\section{Conclusions}

OpenCL can be used to describe approximate operations on neural networks through
the use of compiler flags and approximate algorithm
implementation. On CNN implementations, OpenCL allows
for high-level implementation of hardware definition with
performance gains and resource usage reductions.

Approximate computing techniques can be used
on CNNs to show their error-tolerance property. 
Iteration skipping and memoization 
on pooling layers had
the best results in terms of performance with a minimal accuracy loss.

One of the biggest advantages of using FPGA is being able to properly control
the precision of variables being used. Because of this, changing the precision
on arithmetic operations and input values represents significant reductions on resource
usage on FPGA implementations of CNNs.

The best accuracy-performance ratio found was an increase of 5.66\% on the top-1 error
with a performance gain of 4.39\%. This performance can be boosted
by increasing parallelism and maximizing the resource usage
of the FPGA. A combination of techniques could be used to achieve even 
lower execution times with different levels of accuracy.

\section{Recommendations}

The utilization of approximate computing techniques on FPGA-implemented
CNNs can be expanded through the combination of the different types
of techniques shown in this work. The current project's scope can
be expanded to show the maximum performance and resource usage gain
from approximate techniques.

OpenCL represents a good entry into the hardware definition area.
The project, however, could be expanded upon through the use of
low-level hardware definition languages such as Verilog to produce
finely-tuned approximate solutions.

The current project targeted a relatively low power FPGA. The use
of a more powerful platform could lead to better performance
in comparison to a CPU/GPU CNN implementation.