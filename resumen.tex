\chapter*{Abstract}
\thispagestyle{empty}

The use of FPGAs for accelerating Machine Learning applications
is a research area due to the high specialization level of
this type of platforms. Another technique for application
acceleration is approximate computing. The current work shows
approximate computing techniques being used to guide the
design of an approximate CNN implementation on an FPGA
through the use of OpenCL, a framework for process definition
on heterogenous platforms. It is shown how the combination
of both techniques can lead to significant gains on execution
time and resource usage with little losses on accuracy.

\bigskip

\textbf{Keywords:} \thesisKeywordsEN

\clearpage
\chapter*{Resumen}
\thispagestyle{empty}

El uso de FPGAs para la aceleraci\'on de aplicaciones
de aprendizaje de m\'aquina es un área en desarollo
debido al alto nivel de especialización de este tipo de
plataformas. Otra técnica de aceleración de aplicaciones
es la computación aproximada. El presente trabajo muestra
técnicas de computación aproximada siendo utilizadas para
dirigir el diseño de una implementación en FPGA de redes
neuronales convolucionales por medio de OpenCL, una
herramienta para la definición de procesos en plataformas
heterogéneas. Se demuestra cómo la combinación
de ambas técnicas puede llevar a ganancias significativas
en tiempo de ejecución y uso de recursos de la plataforma
con pequeñas pérdidas en exactitud.

\bigskip

\textbf{Palabras clave:} \thesisKeywordsES

\cleardoublepage

%%% Local Variables: 
%%% mode: latex
%%% TeX-master: "main"
%%% End: 
